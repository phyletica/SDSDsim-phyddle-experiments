% \documentclass[]{beamer}
\documentclass[table,aspectratio=169]{beamer}
% \documentclass[table,aspectratio=43]{beamer}
% \documentclass[table,aspectratio=43,handout]{beamer}
\usepackage{beamerthemesplit}
\usetheme{boxes}
\usecolortheme{seahorse}
% \useinnertheme{myboxes}
% \usepackage{amsmath}
% \usepackage[fleqn]{amsmath}
\usepackage{ifthen}
\usepackage{xspace}
\usepackage{multirow}
\usepackage{booktabs}
\usepackage{xcolor}
\usepackage{changepage}
\usepackage{tabu}
\usepackage[compatibility=false]{caption}
\captionsetup[figure]{font=scriptsize, labelformat=empty, textformat=simple, justification=centering, skip=2pt}

\usepackage{hyperref}
% \hypersetup{pdfborder={0 0 0}, colorlinks=true, urlcolor=black, linkcolor=black, citecolor=black}
\hypersetup{pdfborder={0 0 0}, colorlinks=true, urlcolor=blue, linkcolor=blue, citecolor=blue}

\usepackage[bibstyle=../bib/joaks-slides,maxnames=1,firstinits=true,uniquename=init,backend=biber,bibencoding=utf8,date=year]{biblatex}

\newrobustcmd*{\shortfullcite}{\AtNextCite{\renewbibmacro{title}{}\renewbibmacro{in:}{}\renewbibmacro{number}{}}\fullcite}

\newrobustcmd*{\footlessfullcite}{\AtNextCite{\renewbibmacro{title}{}\renewbibmacro{in:}{}}\footfullcite}

\renewbibmacro*{volume+number+eid}{%
    \iffieldundef{volume}
        {}
        {\printfield{volume}}%
}

% Make all footnotes smaller
%\renewcommand{\footnotesize}{\scriptsize}

\setbeamertemplate{blocks}[rounded][shadow=true]

\setbeamercolor{defaultcolor}{bg=structure!30!normal text.bg,fg=black}
\setbeamercolor{block body}{bg=structure!30!normal text.bg,fg=black}
\setbeamercolor{block title}{bg=structure!50!normal text.bg,fg=black}

% Beamer doesn't play well with enumitem, so hacking some list environments
\newenvironment{mydescription}{
    \begin{description}
        \setlength{\leftskip}{-1.5cm}}
    {\end{description}}

\newenvironment{myitemize}[1][]{
    \begin{itemize}[#1]
        \setlength{\leftskip}{-2.5mm}}
    {\end{itemize}}

\newenvironment{tightitemize}[1][]{
    \begin{itemize}[#1]
        \setlength{\leftskip}{-0.9em}
        \setlength{\itemsep}{-0.3ex}}
    {\end{itemize}}

\newenvironment{myenumerate}{
    \begin{enumerate}
        \setlength{\leftskip}{-2.5mm}}
    {\end{enumerate}}


\newenvironment<>{varblock}[2][\textwidth]{%
  \setlength{\textwidth}{#1}
  \begin{actionenv}#3%
    \def\insertblocktitle{#2}%
    \par%
    \usebeamertemplate{block begin}}
  {\par%
    \usebeamertemplate{block end}%
  \end{actionenv}}

\newenvironment{displaybox}[1][\textwidth]
{
    \centerline\bgroup\hfill
    \begin{beamerboxesrounded}[lower=defaultcolor,shadow=true,width=#1]{}
}
{
    \end{beamerboxesrounded}\hfill\egroup
}

\newenvironment{onlinebox}[1][4cm]
{
    \newbox\mybox
    \newdimen\myboxht
    \setbox\mybox\hbox\bgroup%
        \begin{beamerboxesrounded}[lower=defaultcolor,shadow=true,width=#1]{}
    \centering
}
{
    \end{beamerboxesrounded}\egroup
    \myboxht\ht\mybox
    \raisebox{-0.25\myboxht}{\usebox\mybox}\hspace{2pt}
}

% footnote without a marker
\newcommand\barefootnote[1]{%
  \begingroup
  \renewcommand\thefootnote{}\footnote{#1}%
  \addtocounter{footnote}{-1}%
  \endgroup
}

% define formatting for footer
\newcommand{\myfootline}{%
    {\it
    \insertshorttitle
    \hspace*{\fill} 
    \insertshortauthor\ -- \insertshortinstitute
    % \ifx\insertsubtitle\@empty\else, \insertshortsubtitle\fi
    \hspace*{\fill}
    \insertframenumber/\inserttotalframenumber}}

% set up footer
\setbeamertemplate{footline}{%
    \usebeamerfont{structure}
    \begin{beamercolorbox}[wd=\paperwidth,ht=2.25ex,dp=1ex]{frametitle}%
        \Tiny\hspace*{4mm}\myfootline\hspace{4mm}
    \end{beamercolorbox}}

% remove navigation bar
\beamertemplatenavigationsymbolsempty

\makeatletter
    \newenvironment{noheadline}{
        \setbeamertemplate{headline}[default]
        \def\beamer@entrycode{\vspace*{-\headheight}}
    }{}
\makeatother

%% Set up color palettes %%%%%%%%%%%%%%%%%%%%%%%%%%%%%%%%%%%%%%%%%%%%
\definecolor{citescol}{RGB}{194,101,1}
%\definecolor{citescol}{RGB}{73,0,165}
\definecolor{urlscol}{RGB}{0,150,206}
%\definecolor{urlscol}{RGB}{0,107,124}
%\definecolor{linkscol}{RGB}{187,24,0}
\definecolor{linkscol}{RGB}{149,0,207}
%\definecolor{linkscol}{RGB}{73,0,165}
\definecolor{mycol}{RGB}{25,23,191}
\definecolor{outputcol}{RGB}{34,139,34}
\definecolor{tcol}{RGB}{165,0,14}

% Color palette GreenOrange_6 from https://jiffyclub.github.io/palettable/tableau/
\definecolor{pgreen}     {RGB}{50,162,81}
\definecolor{porange}    {RGB}{255,127,15}
\definecolor{pblue}      {RGB}{60,183,204}
\definecolor{pyellow}    {RGB}{255,217,74}
\definecolor{pteal}      {RGB}{57,115,124}
\definecolor{pauburn}    {RGB}{184,90,13}
%%%%%%%%%%%%%%%%%%%%%%%%%%%%%%%%%%%%%%%%%%%%%%%%%%%%%%%%%%%%%%%%%%%%%

\definecolor{dbluestate}    {RGB}{90,122,148}

\usepackage{pgf}
\usepackage{tikz}
\usetikzlibrary{trees,calc,backgrounds,arrows,positioning,automata}

%%%%%%%%%%%%%%%%%%%%%%%%%%%%%%%%%%%%%%%%%%%%%%%%%%%%%%%%%%%%%%%%%%%%%%%%%%%%%%%
% The `conc` macro is the concentration parameter of the Dirichlet process.
% Set it to the desired value here. Delete (or comment out) the `conc` macro to
% leave probability values blank.
\pgfmathsetmacro\conc{0.5}
\pgfmathsetmacro\cconc{10.0}
\pgfmathsetmacro\ccconc{0.5}
\pgfmathsetmacro\disc{0.0}
\pgfmathsetmacro\ddisc{0.5}
\pgfmathsetmacro\dddisc{0.9}
%%%%%%%%%%%%%%%%%%%%%%%%%%%%%%%%%%%%%%%%%%%%%%%%%%%%%%%%%%%%%%%%%%%%%%%%%%%%%%%

\newcommand{\concentration}{\ensuremath{\alpha}\xspace}
\newcommand{\discount}{\ensuremath{d}\xspace}

\newcommand{\pclass}[3]{%
    \ifthenelse{\equal{#1}{}}%
        {}%
        {\fcolorbox{blue!90}{blue!15}{\catformat{#1}}}%
    \ifthenelse{\equal{#2}{}}%
        {}%
        {\fcolorbox{red!90}{red!15}{\catformat{#2}}}%
    \ifthenelse{\equal{#3}{}}%
        {}%
        {\fcolorbox{gray!90}{gray!15}{\catformat{#3}}}%
}
\newcommand{\catformat}[1]{\textsf{\textbf{#1}}}
\newcommand{\pcat}[3]{%
    \textcolor{blue}{\catformat{#1}}%
    \textcolor{red}{\catformat{#2}}%
    \textcolor{black}{\catformat{#3}}%
}
\newcommand{\branchlabel}[1]{\normalsize #1}
\newcommand{\tiplabel}[1]{\hspace{-0.5em}\normalsize #1}

\newcommand{\calcprob}[2]{%
    \ifthenelse{\isundefined{\conc}}%
        {}%
        {\pgfmathparse{(#1/(\conc+1))*(#2/(\conc+2))}\pgfmathprintnumber{\pgfmathresult}}%
}
\newcommand{\ccalcprob}[2]{%
    \ifthenelse{\isundefined{\cconc}}%
        {}%
        {\pgfmathparse{(#1/(\cconc+1))*(#2/(\cconc+2))}\pgfmathprintnumber{\pgfmathresult}}%
}
\newcommand{\cccalcprob}[2]{%
    \ifthenelse{\isundefined{\ccconc}}%
        {}%
        {\pgfmathparse{(#1/(\ccconc+1))*(#2/(\ccconc+2))}\pgfmathprintnumber{\pgfmathresult}}%
}

\newcommand{\getconc}[1]{%
    \ifthenelse{\isundefined{\conc}}%
        {}%
        {\conc}%
}
\newcommand{\getdisc}[1]{%
    \ifthenelse{\isundefined{\disc}}%
        {}%
        {\disc}%
}

% \tikzset{hide on/.code={\only<#1>{\color{fg!20}}}}
\tikzset{hide on/.code={\only<#1>{\color{white}}}}
\tikzset{
    invisible/.style={opacity=0},
    visible on/.style={alt={#1{}{invisible}}},
    alt/.code args={<#1>#2#3}{%
        \alt<#1>{\pgfkeysalso{#2}}{\pgfkeysalso{#3}}
        % \pgfkeysalso doesn't change the path
    },
}


\newcommand{\ifembed}[2]{#1}
% \newcommand{\ifembed}[2]{#2}
% \newcommand{\ifdoublespacing}[2]{#1}
\newcommand{\ifdoublespacing}[2]{#2}
% \newcommand{\iflinenumbers}[2]{#1}
\newcommand{\iflinenumbers}[2]{#2}
% \newcommand{\ifragged}[2]{#1}
\newcommand{\ifragged}[2]{#2}
% \newcommand{\ifblind}[2]{#1}
\newcommand{\ifblind}[2]{#2}

\newcommand{\jedit}[2]{\sout{#1}{\color{red}{#2}}}
% \newcommand{\jedit}[2]{#2}
\newcommand{\jcomment}[1]{({\color{pgreen}{JRO's comment:}} \textbf{\color{pgreen}{#1}})}
% \newcommand{\jrev}[2]{{\color{red}{#2}}}
\newcommand{\jrev}[2]{#2}
% \newcommand{\jrevv}[2]{{\color{red}{#2}}}
\newcommand{\jrevv}[2]{#2}

\newcommand{\citationNeeded}{\textcolor{magenta}{\textbf{[CITATION NEEDED!]}}\xspace}
\newcommand{\tableNeeded}{\textcolor{magenta}{\textbf{[TABLE NEEDED!]}}\xspace}
\newcommand{\figureNeeded}{\textcolor{magenta}{\textbf{[FIGURE NEEDED!]}}\xspace}
\newcommand{\highLight}[1]{\textcolor{magenta}{\MakeUppercase{#1}}}
\newcommand{\lowLight}[1]{\textcolor{pauburn}{#1}}

\newcommand{\fig}{Figure\xspace}
\newcommand{\figs}{Figures\xspace}
\newcommand{\tbl}{Table\xspace}
\newcommand{\tbls}{Tables\xspace}

\newcommand{\datasets}{data sets\xspace}
\newcommand{\dataset}{data set\xspace}

\newcommand{\editorialNote}[1]{\textcolor{red}{[\textit{#1}]}}
\newcommand{\ignore}[1]{}
\newcommand{\addTail}[1]{\textit{#1}.---}
\newcommand{\super}[1]{\ensuremath{^{\textrm{#1}}}}
\newcommand{\sub}[1]{\ensuremath{_{\textrm{#1}}}}
\newcommand{\dC}{\ensuremath{^\circ{\textrm{C}}}}
\newcommand{\tb}{\hspace{2em}}
\newcommand{\tn}{\tabularnewline}
\newcommand{\spp}[1]{\textit{#1}}
\newcommand{\eg}{\textit{e.g.}\xspace}
\newcommand{\Eg}{\textit{E.g.}\xspace}
\newcommand{\ie}{\textit{i.e.}\xspace}
\newcommand{\Ie}{\textit{I.e.}\xspace}

\providecommand{\e}[1]{\ensuremath{\times 10^{#1}}}

\newcommand{\myemph}[1]{\textbf{\textsl{#1}}}

\newcommand{\change}[2]{{\color{red} #2}\xspace}
\newcommand{\thought}[1]{\textcolor{green}{THOUGHT: #1}}

\newcommand{\widthFigure}[5]{\begin{figure}[htbp]
\begin{center}
    \includegraphics[width=#1\textwidth]{#2}
    \captionsetup{#3}
    \caption{#4}
    \label{#5}
    \end{center}
    \end{figure}}

\newcommand{\heightFigure}[5]{\begin{figure}[htbp]
\begin{center}
    \includegraphics[height=#1\textheight]{#2}
    \captionsetup{#3}
    \caption{#4}
    \label{#5}
    \end{center}
    \end{figure}}

\newcommand{\smartFigure}[5]{%
    \begin{figure}[htbp]
        \begin{center}
            \includegraphics[width=\textwidth,height=#1\textheight,keepaspectratio]{#2}
            \captionsetup{#3}
            \caption{#4}
            \label{#5}
        \end{center}
    \end{figure}
}

\newcommand{\mFigure}[4]{\smartFigure{#1}{#2}{listformat=figList}{#3}{#4}\clearpage}
\newcommand{\embedHeightFigure}[4]{\heightFigure{#1}{#2}{listformat=figList}{#3}{#4}}
\newcommand{\embedWidthFigure}[4]{\widthFigure{#1}{#2}{listformat=figList}{#3}{#4}}
\newcommand{\siFigure}[4]{\smartFigure{#1}{#2}{name=Figure S, labelformat=noSpace, listformat=sFigList}{#3}{#4}\clearpage}

\newcommand{\cpp}{\upshape\texttt{C++}\xspace}
\newcommand{\ecoevolity}{\upshape\texttt{ecoevolity}\xspace}
\newcommand{\pycoevolity}{\upshape\texttt{pycoevolity}\xspace}
\newcommand{\phycoeval}{\upshape\texttt{phycoeval}\xspace}
\newcommand{\simphycoeval}{\upshape\texttt{simphycoeval}\xspace}
\newcommand{\sumphycoeval}{\upshape\texttt{sumphycoeval}\xspace}
\newcommand{\revbayes}{\upshape\texttt{RevBayes}\xspace}
\newcommand{\slim}{\upshape\texttt{SLiM}\xspace}

\newcommand{\sarscov}{SARS-CoV-2\xspace}
\newcommand{\covid}{COVID-19\xspace}

\newcommand{\genmodel}{\ensuremath{M_{G}}\xspace}
\newcommand{\bifmodel}{\ensuremath{M_{IB}}\xspace}

\newcommand{\speciationrate}{\ensuremath{\lambda}\xspace}
\newcommand{\burstrate}{\ensuremath{\lambda_{\beta}}\xspace}
\newcommand{\extinctionrate}{\ensuremath{\mu}\xspace}
\newcommand{\samplingprob}{\ensuremath{\rho}\xspace}
\newcommand{\burstprob}{\ensuremath{\beta}\xspace}

\newcommand{\ifTwoArgs}[3]{\ifthenelse{\equal{#1}{}\or\equal{#2}{}}{}{#3}\xspace}
\newcommand{\ifArg}[2]{\ifthenelse{\equal{#1}{}}{}{#2}\xspace}
\newcommand{\given}{\ensuremath{\,|\,}\xspace}
\newcommand{\pr}{\ensuremath{p}}

\newcommand{\tree}{\ensuremath{T}\xspace}
\newcommand{\nTips}{\ensuremath{N}\xspace}
\newcommand{\nmcmcsamples}{\ensuremath{\mathcal{N}}\xspace}
\newcommand{\node}[1]{\ensuremath{t_{#1}}\xspace}
\newcommand{\nodes}{\ensuremath{\boldsymbol{\node{}}}\xspace}
\newcommand{\divTimeSymbol}{\tau}
\newcommand{\divTime}[1]{\ensuremath{\divTimeSymbol_{#1}}\xspace}
\newcommand{\divTimes}{\ensuremath{\boldsymbol{\divTime{}}}\xspace}
\newcommand{\divTimeParentOf}[1]{\ensuremath{y({#1})}\xspace}
\newcommand{\probChangeDimension}{\ensuremath{\psi}\xspace}
\newcommand{\probBreakPolytomy}{\ensuremath{\Upsilon}\xspace}
\newcommand{\nWaysToBreakPolytomy}{\ensuremathmath{k_b}\xspace}
\newcommand{\bellNumber}{\ensuremath{B}\xspace}
\newcommand{\stirling}[2]{\ensuremath{S_2}(#1, #2)\xspace}
\newcommand{\treeClassPriorProb}[1]{\ensuremath{\pi_{\tree}(#1)}\xspace}
\newcommand{\maxSlide}{\ensuremath{\delta_{\divTime{}}}\xspace}
\newcommand{\uniformDeviate}{\ensuremath{u}\xspace}
\newcommand{\probLumpOverProbSplit}{\ensuremath{\gamma_S}\xspace}
\newcommand{\probLumpNeighborOverProbSplitNeighbor}{\ensuremath{\phi_S}\xspace}
\newcommand{\probSplitOverProbLump}{\ensuremath{\gamma_M}\xspace}
\newcommand{\probSplitNeighborOverProbLumpNeighbor}{\ensuremath{\phi_M}\xspace}
\newcommand{\propdens}[1]{\ensuremath{g_{#1}}\xspace}
\newcommand{\multipropdens}[1]{\ensuremath{\boldsymbol{g}_{#1}}\xspace}
\newcommand{\nOf}[2][]{\ensuremath{n_{#1}(#2)}\xspace}
\newcommand{\nDivs}{\ensuremath{\nOf[]{\divTime{}}}\xspace}
\newcommand{\nNodes}{\ensuremath{\nOf{\node{}}}\xspace}
\newcommand{\nTrees}{\ensuremath{\nOf{\tree{}}}\xspace}
% \newcommand{\nSharedDivs}{\ensuremath{s_{\divTime{}}}\xspace}
\newcommand{\nSharedDivs}{\ensuremath{\nOf[s]{\divTime{}}}\xspace}
\newcommand{\nNodesMappedTo}[1]{\ensuremath{\nOf[]{\node{} \mapsto #1}}\xspace}
\newcommand{\nPolyNodesMovingTo}[1]{\ensuremath{\nOf[p]{\node{} \Rightarrow #1}}\xspace}

\newcommand{\nWaysToSplitAllPolytomies}{\ensuremath{\Phi}\xspace}
\newcommand{\probDivTimePartition}{\ensuremath{\Xi}\xspace}
\newcommand{\modelState}{\ensuremath{\Theta}\xspace}

\newcommand{\multiplier}{\ensuremath{m}\xspace}
\newcommand{\proposed}{\ensuremath{^{\prime}}\xspace}
\newcommand{\tuningparameter}{\ensuremath{\lambda}\xspace}
\newcommand{\uniformdeviate}{\ensuremath{u}\xspace}

\newcommand{\rgmurate}{\ensuremath{u}\xspace}
\newcommand{\grmurate}{\ensuremath{v}\xspace}
\newcommand{\murate}[1][]{\ensuremath{\mu_{#1}}\xspace}
\newcommand{\gfreq}[1][]{\ensuremath{\pi_{#1}}\xspace}

\newcommand{\genetree}[1][]{\ensuremath{g_{#1}}\xspace}

\newcommand{\observedallelecount}[1][]{\ensuremath{n_{#1}}\xspace}
\newcommand{\observedredallelecount}[1][]{\ensuremath{r_{#1}}\xspace}

\newcommand{\nodeallelecount}[2]{\ensuremath{n_{#1}^{#2}}}
\newcommand{\noderedallelecount}[2]{\ensuremath{r_{#1}^{#2}}}

\newcommand{\allelecount}[1][]{\ensuremath{\nodeallelecount{#1}{}}\xspace}
\newcommand{\redallelecount}[1][]{\ensuremath{\noderedallelecount{#1}{}}\xspace}

\newcommand{\leafallelecounts}[1][]{\ensuremath{\mathbf{n}_{#1}}\xspace}
\newcommand{\leafredallelecounts}[1][]{\ensuremath{\mathbf{r}_{#1}}\xspace}

\newcommand{\alldata}[1][]{\ensuremath{\mathbf{D}}\xspace}
\newcommand{\nloci}[1][]{\ensuremath{m_{#1}\xspace}}

\newcommand{\epopsize}[1][]{\ensuremath{N_{e}^{#1}}\xspace}
\newcommand{\allepopsizes}{\ensuremath{\boldsymbol{N_{e}}}\xspace}
\newcommand{\alphaOfDivTimeBetaPrior}{\ensuremath{\alpha_\divTimeSymbol}}
\newcommand{\betaOfDivTimeBetaPrior}{\ensuremath{\beta_\divTimeSymbol}}

\newcommand\mybullet{\leavevmode%
\usebeamertemplate{itemize item}\hspace{.5em}}

\makeatletter
\newcommand*{\rom}[1]{\expandafter\@slowromancap\romannumeral #1@}
\makeatother

\newcommand{\blankslide}{{\setbeamercolor{background canvas}{bg=black}
\setbeamercolor{whitetext}{fg=white}
\begin{frame}<handout:0>[plain]
\end{frame}}}

\newcommand{\whiteslide}{
\begin{frame}<handout:0>[plain]
\end{frame}}


\newcommand{\divModel}[1]{\ensuremath{m_{#1}}\xspace}
\newcommand{\divTimeMap}[1]{\ensuremath{T_{#1}}\xspace}
\newcommand{\alignment}[2]{\ensuremath{X_{#1\protect\ifTwoArgs{#1}{#2}{,}#2}}\xspace}
\newcommand{\divTimeMapVector}{\ensuremath{\mathbf{\divTimeMap{}}}\xspace}
\newcommand{\alignmentVector}{\ensuremath{\mathbf{\alignment{}{}}}\xspace}
\newcommand{\allParameters}[1]{\ensuremath{\theta_{#1}}\xspace}

\newcommand{\abctoolbox}{\href{https://github.com/joaks1/abacus}{ABACUS}\xspace}
\newcommand{\dppmsbayes}{\href{https://github.com/joaks1/dpp-msbayes}{dpp-msbayes}\xspace}
\newcommand{\pymsbayes}{\href{https://joaks1.github.io/PyMsBayes/}{PyMsBayes}\xspace}


\newcommand{\nucleotideformat}[1]
{\tiny
    \begin{minipage}[c][1.1ex][c]{1.1ex}
        % \vspace{-0.9ex}
        % \begin{center}
            % \hspace{-0.7ex}
            \centering{#1}
        % \end{center}
    \end{minipage}
}

% \newcommand{\nA}{\textcolor{red}{A}}
% \newcommand{\nC}{\textcolor{green}{C}}
% \newcommand{\nG}{\textcolor{yellow}{G}}
% \newcommand{\nT}{\textcolor{blue}{T}}
\newcommand{\nA}{\nucleotideformat{A}}
\newcommand{\nC}{\nucleotideformat{C}}
\newcommand{\nG}{\nucleotideformat{G}}
\newcommand{\nT}{\nucleotideformat{T}}

%% Dimensions and macros to calc frame content width and height
\newif\ifsidebartheme
\sidebarthemetrue

\newlength\frametextheight
\newlength\headlessframetextheight
\setlength{\frametextheight}{0.88\textheight}
\setlength{\headlessframetextheight}{1\textheight}

\newdimen\smarttextheight
\newdimen\contentheight
\newdimen\contentwidth
\newdimen\contentleft
\newdimen\contentbottom
\makeatletter
\newcommand*{\calculatespace}{%
\contentheight=\paperheight%
\ifx\beamer@frametitle\@empty%
    \setbox\@tempboxa=\box\voidb@x%
  \else%
    \setbox\@tempboxa=\vbox{%
      \vbox{}%
      {\parskip0pt\usebeamertemplate***{frametitle}}%
    }%
    \ifsidebartheme%
      \advance\contentheight by-1em%
    \fi%
  \fi%
\advance\contentheight by-\ht\@tempboxa%
\advance\contentheight by-\dp\@tempboxa%
\advance\contentheight by-\beamer@frametopskip%
\ifbeamer@plainframe%
\contentbottom=0pt%
\else%
\advance\contentheight by-\headheight%
\advance\contentheight by\headdp%
\advance\contentheight by-\footheight%
\advance\contentheight by4pt%
\contentbottom=\footheight%
\advance\contentbottom by-4pt%
\fi%
\contentwidth=\paperwidth%
\ifbeamer@plainframe%
\contentleft=0pt%
\else%
\advance\contentwidth by-\beamer@rightsidebar%
\advance\contentwidth by-\beamer@leftsidebar\relax%
\contentleft=\beamer@leftsidebar%
\fi%
\smarttextheight=\contentheight%
\advance\smarttextheight by-4mm%
\ifx\beamer@frametitle\@empty%
    \advance\smarttextheight by-3mm%
\fi%
}
\makeatother

\newcommand{\smartgraphic}[3][\frametextheight]{%
\includegraphics#2[width=\linewidth,height=#1,keepaspectratio]{#3}%
}

% \bibliography{../bib/references}
\bibliography{references}
\usepackage{lmodern}
\usepackage{bm}
\usepackage{pdfpages}

\usepackage{appendixnumberbeamer}
% \beamerdefaultoverlayspecification{<+->}

% \ExecuteBibliographyOptions{maxnames=3}

\setbeamertemplate{footline}{}
\renewcommand\footnoterule{}
\setbeamertemplate{footnote}{%
  \parindent 0em\noindent%
  \raggedleft
  \usebeamercolor{footnote}\hbox to 0.8em{\hfil\insertfootnotemark}\insertfootnotetext\par%
}

\title[Trait-dependent shared divergences]{Exploring trait-dependent models of multi-lineage processes of diversification with machine learning}

\author[Jamie Oaks]{
    Jamie R.\ Oaks\inst{1}
}
\institute[\href{https://phyletica.org}{phyletica.org}]{
    \inst{1}%
    Museum of Natural History \& Department of Biology, Auburn
    University
}

% \date{\today}
\date{June 23, 2025}

% \setbeamersize{text margin left=3mm, text margin right=3mm}
\setbeamersize{text margin left=5mm, text margin right=5mm}

\begin{document}

\begin{frame}[t]
    \begin{columns}[c]
        \column{.51\textwidth}

        \begin{minipage}[c][\headlessframetextheight][c]{\columnwidth}
        \begin{center}

            \begin{displaybox}[0.88\linewidth]
            \begin{center}
                \begin{minipage}[c][0.22\textheight][c]{0.88\linewidth}
                \begin{center}
                    \large Exploring trait-dependent models of multi-lineage
                    processes of diversification with machine learning
                \end{center}
                \end{minipage}
            \end{center}
            \end{displaybox}

            \vspace{2.5ex}
            \setlength{\tabcolsep}{0.15em}
            \begin{tabular}{@{}cc@{}}
                \multicolumn{2}{c}{\large\bf Jamie Oaks} \\
                \multicolumn{2}{c}{Auburn University} \\ % Museum of Natural History \\
                \multicolumn{2}{c}{\href{https://phyletica.org}{phyletica.org}} \\
                \href{https://github.com/joaks1}{\includegraphics[height=2.5ex]{../images/github-logo.png}}
                & \href{https://github.com/joaks1}{joaks1} \& \href{https://github.com/phyletica}{phyletica} \\
                \href{https://bsky.app/profile/jamieoaks.bsky.social}{\includegraphics[height=2.5ex]{../images/bluesky-logo.png}}
                & \href{https://bsky.app/profile/jamieoaks.bsky.social}{{@}jamieoaks.bsky.social} \\
                % Slides: \href{https://phyletica.org/slides/evol2025.pdf}{phyletica.org/slides/evol2025.pdf} 
            \end{tabular}

            \vspace{-1.0ex}
            \begin{center}
                \href{https://phyletica.org/slides/evol2025.pdf}{%
                \includegraphics[height=0.29\headlessframetextheight]{../images/qr-code-evol2025.png}}
                \captionof{figure}{\href{https://phyletica.org/slides/evol2025.pdf}{\small phyletica.org/slides/evol2025.pdf}}
            \end{center}

        \end{center}
        \end{minipage}

        \column{.49\textwidth}

        \begin{minipage}[t][\headlessframetextheight][b]{\columnwidth}
            \begin{figure}
                \begin{center}
                    \includegraphics[width=\textwidth,height=\headlessframetextheight,keepaspectratio]{../images/darwin-tol-copyright-boris-kulikov-2007.jpg}
                    \vspace{-1.0mm}
                    \caption{\tiny \copyright~2007 Boris Kulikov \href{https://boris-kulikov.blogspot.com/}{boris-kulikov.blogspot.com}}
                \end{center}
            \end{figure}
        \end{minipage}

    \end{columns}

\end{frame}


\section{An assumption (i.e., exciting opportunity) in phylogenetics}
% We've long assumed divergences are independent

% \begin{frame}[t]
    \begin{columns}

        \column{0.52\textwidth}

        \begin{minipage}[c][\frametextheight][c]{0.95\columnwidth}
        \begin{myitemize}
            \item<2-> \textbf{Assumption:} All processes of diversification
                affect each lineage independently
                \vspace{3mm}
        \end{myitemize}
        \end{minipage}

        \column{0.46\textwidth}

        \begin{minipage}[c][\frametextheight][c]{\columnwidth}
            \centering
            \smartgraphic{<1->}{../images/trees/pair-trees/tree-3-bare-crop.pdf}
        \end{minipage}
    \end{columns}
    \barefootnote{\tiny\href{https://phylopic.org/}{phylopic.org CC0}}
\end{frame}



% Show example of how assumption is violated

\begin{frame}
    %% This is a tikz file
\tikzset{node lower left/.style={font=\small,anchor=north east,text height=0.240cm,text depth=0.068cm,inner sep=0.03cm},
leaf/.style={font=\small,anchor=west,text height=0.240cm,text depth=0.068cm},
node upper left/.style={font=\small,anchor=south east,text height=0.240cm,text depth=0.068cm,inner sep=0.03cm},
bracket label/.style={font=\small,anchor=west,text height=0.240cm,text depth=0.068cm,inner sep=0.1cm},
node upper right/.style={font=\small,anchor=south west,text height=0.240cm,text depth=0.068cm,inner sep=0.03cm},
node right/.style={font=\small,anchor=west,text height=0.240cm,text depth=0.068cm,inner sep=0.03cm},
branch/.style={font=\tiny,text height=0.144cm,text depth=0.041cm,inner sep=0.025cm},
root/.style={font=\small,anchor=east,text height=0.240cm,text depth=0.068cm},
node lower right/.style={font=\small,anchor=north west,text height=0.240cm,text depth=0.068cm,inner sep=0.03cm}}

\centering{
\resizebox{!}{\frametextheight}{%
\begin{tikzpicture}[ultra thick,inner sep=0.1cm]
%  4:\hspace{20mm}\includegraphics[width=15mm]{../images/phylopics/gekko-vittatus-4-yellow-shadow.png}
%  3
% +2:\hspace{20mm}\includegraphics[width=15mm]{../images/phylopics/gekko-vittatus-5-teal-shadow.png}
% ||
% |6:\hspace{20mm}\includegraphics[width=15mm]{../images/phylopics/gekko-vittatus-6-auburn-shadow.png}
% |
% |10:\hspace{20mm}\includegraphics[width=15mm]{../images/phylopics/gecko-pixabay-cc0-4-yellow-shadow.png}
% 09
% |81:\hspace{20mm}\includegraphics[width=15mm]{../images/phylopics/gecko-pixabay-cc0-5-teal-shadow.png}
% ||
% |12:\hspace{20mm}\includegraphics[width=15mm]{../images/phylopics/gecko-pixabay-cc0-6-auburn-shadow.png}
% 7
% |15:\hspace{20mm}\includegraphics[width=15mm]{../images/phylopics/gekko-gecko-4-yellow-shadow.png}
% |14
% +13:\hspace{20mm}\includegraphics[width=15mm]{../images/phylopics/gekko-gecko-5-teal-shadow.png}
%  |
%  17:\hspace{20mm}\includegraphics[width=15mm]{../images/phylopics/gekko-gecko-6-auburn-shadow.png}

% The scale is 1.000000, and the yScale is 0.800000

%% Coordinates of nodes.
\coordinate (island) at (11.6, 3.2);
\node[visible on=<1>] at (island) {\includegraphics[height=8.0cm]{../images/islands-1.pdf}};
\node[visible on=<2->] at (island) {\includegraphics[height=8.0cm]{../images/islands-2.pdf}};
% \node[visible on=<3->] at (island) {\includegraphics[height=8.0cm]{../images/islands-3.pdf}};

% \draw [visible on=<3->, pteal,very thick,dashed] (6.600,-0.300) -- (6.600,6.728);
% \node [font=\LARGE,yshift=-0.200cm,text height=0.415cm,text depth=0.118cm] at (6.600,-0.300) {\color{pteal}$\tau_{\scriptscriptstyle 1}$};
\draw [visible on=<2->, pauburn,very thick,dashed] (4.600,-0.300) -- (4.600,6.728);
% \node [font=\LARGE,yshift=-0.200cm,text height=0.415cm,text depth=0.118cm] at (4.600,-0.300) {\color{pauburn}$\tau_{\scriptscriptstyle 2}$};
\coordinate (n0) at (0.000,3.600);
\coordinate (n1) at (0.100,3.600);
\coordinate (n1p) at (0.000,3.600);
\coordinate (n2) at (4.600,5.400);
\coordinate (n2h) at (3.200,5.400);
\coordinate (n2p) at (0.100,5.400);
\coordinate (n3) at (6.600,6.000);
\coordinate (n3p) at (4.600,6.000);
\coordinate (n3h) at (5.500,6.000);
\coordinate (n4) at (7.100,6.400);
\coordinate (n4p) at (6.600,6.400);
\coordinate (n5) at (7.100,5.600);
\coordinate (n5p) at (6.600,5.600);
\coordinate (n6) at (7.100,4.800);
\coordinate (n6p) at (4.600,4.800);
\coordinate (n6h) at (5.500,4.800);
\coordinate (n7) at (1.100,1.800);
\coordinate (n7p) at (0.100,1.800);
\coordinate (n8) at (4.600,3.000);
\coordinate (n8h) at (3.200,3.000);
\coordinate (n8p) at (1.100,3.000);
\coordinate (n9) at (6.600,3.600);
\coordinate (n9p) at (4.600,3.600);
\coordinate (n9h) at (5.500,3.600);
\coordinate (n10) at (7.100,4.000);
\coordinate (n10p) at (6.600,4.000);
\coordinate (n11) at (7.100,3.200);
\coordinate (n11p) at (6.600,3.200);
\coordinate (n12) at (7.100,2.400);
\coordinate (n12p) at (4.600,2.400);
\coordinate (n12h) at (5.500,2.400);
\coordinate (n13) at (4.600,0.600);
\coordinate (n13h) at (3.200,0.600);
\coordinate (n13p) at (1.100,0.600);
\coordinate (n14) at (6.600,1.200);
\coordinate (n14p) at (4.600,1.200);
\coordinate (n14h) at (5.500,1.200);
\coordinate (n15) at (7.100,1.600);
\coordinate (n15p) at (6.600,1.600);
\coordinate (n16) at (7.100,0.800);
\coordinate (n16p) at (6.600,0.800);
\coordinate (n17) at (7.100,0.000);
\coordinate (n17p) at (4.600,0.000);
\coordinate (n17h) at (5.500,0.000);

%% horizontal lines
\draw [visible on=<1->] (n1p) -- (n1);
\draw [visible on=<1>] (n2p) -- (n2h);
\draw [visible on=<2->] (n2p) -- (n2);
% \draw [visible on=<3->] (n3p) -- (n3);
\draw [visible on=<2->] (n3p) -- (n3h);
% \draw [visible on=<3->] (n4p) -- (n4);
% \draw [visible on=<3->] (n5p) -- (n5);
% \draw [visible on=<3->] (n6p) -- (n6);
\draw [visible on=<2->] (n6p) -- (n6h);
\draw [visible on=<1->] (n7p) -- (n7);
\draw [visible on=<1>] (n8p) -- (n8h);
\draw [visible on=<2->] (n8p) -- (n8);
% \draw [visible on=<3->] (n9p) -- (n9);
\draw [visible on=<2->] (n9p) -- (n9h);
% \draw [visible on=<3->] (n10p) -- (n10);
% \draw [visible on=<3->] (n11p) -- (n11);
% \draw [visible on=<3->] (n12p) -- (n12);
\draw [visible on=<2->] (n12p) -- (n12h);
\draw [visible on=<1>] (n13p) -- (n13h);
\draw [visible on=<2->] (n13p) -- (n13);
% \draw [visible on=<3->] (n14p) -- (n14);
\draw [visible on=<2->] (n14p) -- (n14h);
% \draw [visible on=<3->] (n15p) -- (n15);
% \draw [visible on=<3->] (n16p) -- (n16);
% \draw [visible on=<3->] (n17p) -- (n17);
\draw [visible on=<2->] (n17p) -- (n17h);

%% vertical lines
\draw [visible on=<1->,line cap=rect] (n2p) -- (n7p);
\draw [visible on=<2->,line cap=rect] (n3p) -- (n6p);
% \draw [visible on=<3->,line cap=rect] (n4p) -- (n5p);
\draw [visible on=<1->,line cap=rect] (n8p) -- (n13p);
\draw [visible on=<2->,line cap=rect] (n9p) -- (n12p);
% \draw [visible on=<3->,line cap=rect] (n10p) -- (n11p);
\draw [visible on=<2->,line cap=rect] (n14p) -- (n17p);
% \draw [visible on=<3->,line cap=rect] (n15p) -- (n16p);

%% leaf labels
\node [visible on=<1>] at (n2h) {\hspace{20mm}\includegraphics[width=15mm]{../images/phylopics/gekko-vittatus-5-teal-shadow.png}};
\node [visible on=<1>] at (n8h) {\hspace{20mm}\includegraphics[width=15mm]{../images/phylopics/gecko-pixabay-cc0-5-teal-shadow.png}};
\node [visible on=<1>] at (n13h) {\hspace{20mm}\includegraphics[width=15mm]{../images/phylopics/gekko-gecko-5-teal-shadow.png}};

\node [visible on=<2->] at (n3h) {\hspace{20mm}\includegraphics[width=15mm]{../images/phylopics/gekko-vittatus-5-teal-shadow.png}};
\node [visible on=<2->] at (n6h) {\hspace{20mm}\includegraphics[width=15mm]{../images/phylopics/gekko-vittatus-6-auburn-shadow.png}};
\node [visible on=<2->] at (n9h) {\hspace{20mm}\includegraphics[width=15mm]{../images/phylopics/gecko-pixabay-cc0-5-teal-shadow.png}};
\node [visible on=<2->] at (n12h) {\hspace{20mm}\includegraphics[width=15mm]{../images/phylopics/gecko-pixabay-cc0-6-auburn-shadow.png}};
\node [visible on=<2->] at (n14h) {\hspace{20mm}\includegraphics[width=15mm]{../images/phylopics/gekko-gecko-5-teal-shadow.png}};
\node [visible on=<2->] at (n17h) {\hspace{20mm}\includegraphics[width=15mm]{../images/phylopics/gekko-gecko-6-auburn-shadow.png}};

% \node [visible on=<3->] at (n4) {\hspace{20mm}\includegraphics[width=15mm]{../images/phylopics/gekko-vittatus-5-teal-shadow.png}};
% \node [visible on=<3->] at (n5) {\hspace{20mm}\includegraphics[width=15mm]{../images/phylopics/gekko-vittatus-4-yellow-shadow.png}};
% \node [visible on=<3->] at (n6) {\hspace{20mm}\includegraphics[width=15mm]{../images/phylopics/gekko-vittatus-6-auburn-shadow.png}};
% \node [visible on=<3->] at (n10) {\hspace{20mm}\includegraphics[width=15mm]{../images/phylopics/gecko-pixabay-cc0-5-teal-shadow.png}};
% \node [visible on=<3->] at (n11) {\hspace{20mm}\includegraphics[width=15mm]{../images/phylopics/gecko-pixabay-cc0-4-yellow-shadow.png}};
% \node [visible on=<3->] at (n12) {\hspace{20mm}\includegraphics[width=15mm]{../images/phylopics/gecko-pixabay-cc0-6-auburn-shadow.png}};
% \node [visible on=<3->] at (n15) {\hspace{20mm}\includegraphics[width=15mm]{../images/phylopics/gekko-gecko-5-teal-shadow.png}};
% \node [visible on=<3->] at (n16) {\hspace{20mm}\includegraphics[width=15mm]{../images/phylopics/gekko-gecko-4-yellow-shadow.png}};
% \node [visible on=<3->] at (n17) {\hspace{20mm}\includegraphics[width=15mm]{../images/phylopics/gekko-gecko-6-auburn-shadow.png}};

\end{tikzpicture}
}
}

    \barefootnote{\tiny\href{https://phylopic.org/}{phylopic.org CC0}}
\end{frame}

\begin{frame}[c,label=violations]
    % \frametitle{Violations are pervasive and interesting}
    % \frametitle{Shared divergences are pervasive and interesting}

    \begin{columns}

        \column{0.49\textwidth}

        \begin{minipage}[c][\frametextheight][c]{0.95\columnwidth}
            \raggedright
            % \uncover<1->{
            \textbf{Biogeography}
            \begin{itemize}
                \item Environmental changes that affect whole communities of species
            \end{itemize}
            % }

            % \uncover<2->{
            \textbf{Epidemiology}
            \begin{itemize}
                % \item Disease spread via co-infected individuals
                \item Transmission at social gatherings
            \end{itemize}
            % }

            % \uncover<3->{
            \textbf{Genome evolution}
            \begin{itemize}
                \item Duplication of a chromosome segment harboring a gene family
            \end{itemize}
            % }

            % \uncover<4->{
            % \textbf{Endosymbiont evolution} (e.g., parasites, microbiome)
            % \begin{itemize}
            %     \item Speciation of the host
            %     \item Co-colonization of new host species
            % \end{itemize}
            % }

            \bigskip
            \uncover<2->{
                \textbf{\textit{These violations are interesting!}}
            }
        \end{minipage}

        \column{0.46\textwidth}

        \begin{minipage}[c][\frametextheight][c]{\columnwidth}
            \centering
            % \resizebox{!}{0.75\frametextheight}{%
            % \resizebox{!}{0.71\frametextheight}{%
            \resizebox{0.8\columnwidth}{!}{%
                %% This is a tikz file
\tikzset{node lower left/.style={font=\small,anchor=north east,text height=0.240cm,text depth=0.068cm,inner sep=0.03cm},
leaf/.style={font=\small,anchor=west,text height=0.240cm,text depth=0.068cm},
node upper left/.style={font=\small,anchor=south east,text height=0.240cm,text depth=0.068cm,inner sep=0.03cm},
bracket label/.style={font=\small,anchor=west,text height=0.240cm,text depth=0.068cm,inner sep=0.1cm},
node upper right/.style={font=\small,anchor=south west,text height=0.240cm,text depth=0.068cm,inner sep=0.03cm},
node right/.style={font=\small,anchor=west,text height=0.240cm,text depth=0.068cm,inner sep=0.03cm},
branch/.style={font=\tiny,text height=0.144cm,text depth=0.041cm,inner sep=0.025cm},
root/.style={font=\small,anchor=east,text height=0.240cm,text depth=0.068cm},
node lower right/.style={font=\small,anchor=north west,text height=0.240cm,text depth=0.068cm,inner sep=0.03cm}}

\begin{tikzpicture}[ultra thick,inner sep=0.1cm]
%  4:\hspace{20mm}\includegraphics[width=15mm,resolution=150]{../images/phylopics/gekko-vittatus-4-yellow-shadow.png}
%  3
% +2:\hspace{20mm}\includegraphics[width=15mm,resolution=150]{../images/phylopics/gekko-vittatus-5-teal-shadow.png}
% ||
% |6:\hspace{20mm}\includegraphics[width=15mm,resolution=150]{../images/phylopics/gekko-vittatus-6-auburn-shadow.png}
% |
% |10:\hspace{20mm}\includegraphics[width=15mm,resolution=150]{../images/phylopics/gecko-pixabay-cc0-4-yellow-shadow.png}
% 09
% |81:\hspace{20mm}\includegraphics[width=15mm,resolution=150]{../images/phylopics/gecko-pixabay-cc0-5-teal-shadow.png}
% ||
% |12:\hspace{20mm}\includegraphics[width=15mm,resolution=150]{../images/phylopics/gecko-pixabay-cc0-6-auburn-shadow.png}
% 7
% |15:\hspace{20mm}\includegraphics[width=15mm,resolution=150]{../images/phylopics/gekko-gecko-4-yellow-shadow.png}
% |14
% +13:\hspace{20mm}\includegraphics[width=15mm,resolution=150]{../images/phylopics/gekko-gecko-5-teal-shadow.png}
%  |
%  17:\hspace{20mm}\includegraphics[width=15mm,resolution=150]{../images/phylopics/gekko-gecko-6-auburn-shadow.png}

% The scale is 1.000000, and the yScale is 0.800000

% \draw [pteal,very thick,dashed] (6.600,-0.300) -- (6.600,6.728);
% \node [font=\LARGE,yshift=-0.200cm,text height=0.415cm,text depth=0.118cm] at (6.600,-0.300) {\color{pteal}$\tau_{\scriptscriptstyle 1}$};
\draw [pauburn,very thick,dashed] (4.600,-0.300) -- (4.600,6.728);
% \node [font=\LARGE,yshift=-0.200cm,text height=0.415cm,text depth=0.118cm] at (4.600,-0.300) {\color{pauburn}$\tau_{\scriptscriptstyle 2}$};
\coordinate (n0) at (0.000,3.600);
\coordinate (n1) at (0.100,3.600);
\coordinate (n1p) at (0.000,3.600);
\coordinate (n2) at (4.600,5.400);
\coordinate (n2h) at (3.200,5.400);
\coordinate (n2p) at (0.100,5.400);
\coordinate (n3) at (6.600,6.000);
\coordinate (n3p) at (4.600,6.000);
\coordinate (n3h) at (5.500,6.000);
\coordinate (n4) at (7.100,6.400);
\coordinate (n4p) at (6.600,6.400);
\coordinate (n5) at (7.100,5.600);
\coordinate (n5p) at (6.600,5.600);
\coordinate (n6) at (7.100,4.800);
\coordinate (n6p) at (4.600,4.800);
\coordinate (n6h) at (5.500,4.800);
\coordinate (n7) at (1.100,1.800);
\coordinate (n7p) at (0.100,1.800);
\coordinate (n8) at (4.600,3.000);
\coordinate (n8h) at (3.200,3.000);
\coordinate (n8p) at (1.100,3.000);
\coordinate (n9) at (6.600,3.600);
\coordinate (n9p) at (4.600,3.600);
\coordinate (n9h) at (5.500,3.600);
\coordinate (n10) at (7.100,4.000);
\coordinate (n10p) at (6.600,4.000);
\coordinate (n11) at (7.100,3.200);
\coordinate (n11p) at (6.600,3.200);
\coordinate (n12) at (7.100,2.400);
\coordinate (n12p) at (4.600,2.400);
\coordinate (n12h) at (5.500,2.400);
\coordinate (n13) at (4.600,0.600);
\coordinate (n13h) at (3.200,0.600);
\coordinate (n13p) at (1.100,0.600);
\coordinate (n14) at (6.600,1.200);
\coordinate (n14p) at (4.600,1.200);
\coordinate (n14h) at (5.500,1.200);
\coordinate (n15) at (7.100,1.600);
\coordinate (n15p) at (6.600,1.600);
\coordinate (n16) at (7.100,0.800);
\coordinate (n16p) at (6.600,0.800);
\coordinate (n17) at (7.100,0.000);
\coordinate (n17p) at (4.600,0.000);
\coordinate (n17h) at (5.500,0.000);

%% horizontal lines
\draw (n1p) -- (n1);
\draw (n2p) -- (n2h);
\draw (n2p) -- (n2);
% \draw (n3p) -- (n3);
\draw (n3p) -- (n3h);
% \draw (n4p) -- (n4);
% \draw (n5p) -- (n5);
% \draw (n6p) -- (n6);
\draw (n6p) -- (n6h);
\draw (n7p) -- (n7);
\draw (n8p) -- (n8h);
\draw (n8p) -- (n8);
% \draw (n9p) -- (n9);
\draw (n9p) -- (n9h);
% \draw (n10p) -- (n10);
% \draw (n11p) -- (n11);
% \draw (n12p) -- (n12);
\draw (n12p) -- (n12h);
\draw (n13p) -- (n13h);
\draw (n13p) -- (n13);
% \draw (n14p) -- (n14);
\draw (n14p) -- (n14h);
% \draw (n15p) -- (n15);
% \draw (n16p) -- (n16);
% \draw (n17p) -- (n17);
\draw (n17p) -- (n17h);

%% vertical lines
\draw [line cap=rect] (n2p) -- (n7p);
\draw [line cap=rect] (n3p) -- (n6p);
% \draw [line cap=rect] (n4p) -- (n5p);
\draw [line cap=rect] (n8p) -- (n13p);
\draw [line cap=rect] (n9p) -- (n12p);
% \draw [line cap=rect] (n10p) -- (n11p);
\draw [line cap=rect] (n14p) -- (n17p);
% \draw [line cap=rect] (n15p) -- (n16p);

\end{tikzpicture}

            }
        \end{minipage}
    \end{columns}

\end{frame}


% Progress toward that assumption
%   Phycoeval
%   BD with share divergences (Magee and Hohna preprint 2021)
%   We have phylo inference machinery and BD likelihood of trees

\setcounter{footnote}{0}
\begin{frame}[t]
    \frametitle{Progress on shared divergences: Phycoeval}

    \begin{columns}

        \column{0.55\textwidth}
        \begin{minipage}[c][\frametextheight][c]{\columnwidth}
            \begin{center}
                \includegraphics[width=\columnwidth,height=0.13\frametextheight,keepaspectratio]{../images/phycoeval-logo.pdf}

                \smallskip
                \includegraphics[width=\columnwidth,height=0.26\frametextheight,keepaspectratio]{../images/rj-moves-merge-up-cropped.pdf}
            \end{center}

            \smallskip
            \begin{itemize}
                \item Bayesian inference of trees with shared divergences
            \end{itemize}
        \end{minipage}

        \column{0.44\textwidth}

        \begin{minipage}[c][\frametextheight][t]{\columnwidth}
            \begin{center}
                \includegraphics[width=\columnwidth,height=\frametextheight,keepaspectratio]{../images/cyrt-phycoeval-cam-tree-only.pdf}
            \end{center}
        \end{minipage}

    \end{columns}

    \barefootnote{\tiny\shortfullcite{Oaks2021phycoevalshort}}
\end{frame}
\setcounter{footnote}{0}

\begin{frame}[t]
    \frametitle{Progress on shared divergences: Birth-Death + Bursts}

    \begin{columns}

        \column{0.49\textwidth}
        \begin{minipage}[c][\frametextheight][c]{\columnwidth}
            \begin{itemize}
                \item Andrew Magee \& Sebastian H\"{o}hna
                \item Likelihood of tree under BD model with ``burst'' events
            \end{itemize}
        \end{minipage}

        \column{0.49\textwidth}

        \begin{minipage}[c][\frametextheight][t]{\columnwidth}
            \begin{center}
                \includegraphics[width=\columnwidth,height=0.45\frametextheight,keepaspectratio]{../images/magee-hohna-2021-burst-extinction.png}

                \smallskip
                \includegraphics[width=\columnwidth,height=0.52\frametextheight,keepaspectratio]{../images/magee-hohna-2021-bd-model.png}
            \end{center}
        \end{minipage}

    \end{columns}

    \barefootnote{\tiny\shortfullcite{Magee2021}}
\end{frame}


% But I want to know how/if lineage traits influence propensity for divergence events
%   Do geckos inhabiting shallow-water islands have higher rates of shared divergences?
%   Were geckos adapted to limestone karst more prone to diversification via fragmentation of karst formations?
%   Do strains of virus differ in their ability to spread at social gatherings?

\begin{frame}[t]

%   Do geckos inhabiting shallow-water islands have higher rates of shared divergences?
%   Were geckos adapted to limestone karst more prone to diversification via fragmentation of karst formations?
%   Do strains of virus differ in their ability to spread at social gatherings?

    \begin{minipage}[t][0.92\headlessframetextheight][t]{\textwidth}
        \begin{minipage}[t][0.9\headlessframetextheight][t]{0.54\textwidth}
            \begin{itemize}
                \item<1-> Did geckos adapted to karst diversify via fragmentation of karst formations?
            \end{itemize}
            \begin{center}
                \includegraphics[width=0.7\columnwidth,height=0.8\headlessframetextheight,keepaspectratio]{../images/myanmar/PANO_20181106_165228.vr.jpg}

                \smallskip
                \includegraphics[width=0.7\columnwidth,height=0.8\headlessframetextheight,keepaspectratio]{../images/karst-cyrts/UNADJUSTEDRAW_thumb_ac4e.jpg}
            \end{center}
        \end{minipage}
        \begin{minipage}[t][0.9\headlessframetextheight][t]{0.45\textwidth}
            \begin{itemize}
                \item<2-> Do viral strains differ in ability to spread at social gatherings?
            \end{itemize}

            \vspace{1.4cm}
            \begin{center}
                \uncover<2->{
                \includegraphics[width=0.45\columnwidth,height=0.3\headlessframetextheight,keepaspectratio]{../images/phylopics/podoviridae-keesey-pub-domain-5-teal-shadow.png}
                \hspace{5mm}
                \includegraphics[width=0.45\columnwidth,height=0.3\headlessframetextheight,keepaspectratio]{../images/phylopics/podoviridae-keesey-pub-domain-6-auburn-shadow.png}
                }
            \end{center}
        \end{minipage}

    \end{minipage}
    \begin{minipage}[t][0.08\headlessframetextheight][t]{\textwidth}
        \begin{center}
            \uncover<3->{
                \textbf{\textit{These questions require state-dependent models}}
            }
        \end{center}
    \end{minipage}

    \vspace{-5mm}
    \barefootnote{\tiny\href{https://phylopic.org/}{phylopic.org CC0}}
\end{frame}


% Walk through state-dependent multi-lineage BD model
% Coded simulator to explore

{
\setbeamertemplate{headline}{\vspace{1.3cm}\hspace{3mm}\includegraphics[height=0.28\frametextheight]{../images/qr-code-sdsdsim-with-name.png}\hfill}
\begin{frame}[t]
    \frametitle{State-dependent model of shared divergences}

    \vspace{-2mm}
    \begin{columns}[T]
        \column{.42\textwidth}

        \begin{minipage}[t][\frametextheight][t]{\columnwidth}
            \begin{figure}
                \begin{center}
                    \includegraphics[width=\textwidth,height=\frametextheight,keepaspectratio]{../images/sdsd-model-example.pdf}
                \end{center}
            \end{figure}
        \end{minipage}

        \column{.01\textwidth}

        \column{.55\textwidth}

        \begin{minipage}[t][\frametextheight][c]{\columnwidth}

            Parameters:

            \begin{itemize}
                \item Transition rates between character \\
                    \textbf{State 0} $\leftrightarrows$ \textbf{\textcolor{dbluestate}{State 1}}
                \item Speciation rate, $\lambda$ (per-lineage)
                \item Extinction rate, $\mu$ (per-lineage)
                \item Rate of ``burst events'', $\beta$ (tree-wide)
                \item \textbf{State 0} probability of speciation at events, $p_0$
                \item \textbf{\textcolor{dbluestate}{State 1}} probability of speciation at events, $p_1$
                \begin{itemize}
                    \item<2-> $p_1 \beta$ = Per-lineage expected rate of burst divergences
                \end{itemize}
            \end{itemize}

            % \bigskip
            % \uncover<2->{
            % Can get per-lineage expected rate of burst divergences
            % \begin{itemize}
            %     \item \Eg, \hspace{0.2em} $p_0 \beta$
            % \end{itemize}
            % }

            % \medskip
            % \uncover<3->{
            % Ideally, full Bayesian inference of state-dependent burst model and
            % tree
            % }

            % \medskip
            % \uncover<4->{
            % Is all that work worth it?
            % }

            % \vspace{-4mm}
            % \uncover<3->{
            % \begin{center}
            %     \href{https://github.com/phyletica/SDSDsim}{%
            %     \includegraphics[height=0.25\frametextheight]{../images/qr-code-SDSDsim.png}}
            %     \captionof{figure}{\href{https://github.com/phyletica/SDSDsim}{\normalsize \textbf{\textit{SDSDsim}}}}
            % \end{center}
            % }

        \end{minipage}

    \end{columns}

\end{frame}
}


% {
% \setbeamertemplate{headline}{\vspace{0.1cm}\hfill\includegraphics[height=0.25\frametextheight]{../images/qr-code-sdsdsim-with-name.png}$\,\,\,$}
\begin{frame}
    \frametitle{Some random example trees}

    \begin{center}
        \includegraphics[width=0.9\textwidth]{../images/bd-sim-trees-labelled.pdf}

        Both models have same expected rates of state-dependent diversification

    \end{center}

\end{frame}
% }


% Sounds cool, but is it worth applying for funds to spend years developing these models and inference machinery?
%   Is there information in the trees produced by these models to detect shared div rates and differences based on lineage char states?

% Could try to prove identifiability, but practical interpretation is difficult

\begin{frame}[t]
    \frametitle{What next?}

    \vspace{-2mm}
    \begin{columns}[T]
        \column{.42\textwidth}

        \begin{minipage}[t][\frametextheight][t]{\columnwidth}
            \begin{figure}
                \begin{center}
                    \includegraphics[width=\textwidth,height=\frametextheight,keepaspectratio]{../images/sdsd-model-example.pdf}
                \end{center}
            \end{figure}
        \end{minipage}

        \column{.01\textwidth}

        \column{.55\textwidth}

        \begin{minipage}[c][\frametextheight][c]{\columnwidth}

            Ideally, full Bayesian inference of state-dependent burst model and
            tree

            \bigskip
            \uncover<2->{
            That's a lot of work\ldots
            }

            \bigskip
            \uncover<3->{
            Is it worth it?
            \begin{itemize}
                \item Is there information in the trees to detect shared divergence rates and state-dependent differences?
            \end{itemize}
            }

        \end{minipage}

    \end{columns}

\end{frame}


% Machine-learning seems like a great way to learn patterns in data
%   Phyddle makes it easy!

\begin{frame}
    \begin{columns}

        \column{0.49\textwidth}
        \begin{minipage}[c][\frametextheight][c]{\columnwidth}
            \begin{itemize}
                \item User-friendly deep learning with trees
                \item Ammon Thompson \& Michael Landis 
            \end{itemize}
        \end{minipage}

        \column{0.49\textwidth}

        \begin{minipage}[c][\frametextheight][t]{\columnwidth}
            \begin{center}
                \includegraphics[width=\columnwidth,height=\frametextheight,keepaspectratio]{../images/phyddle_pipeline_with_title.png}
            \end{center}
        \end{minipage}

    \end{columns}

    \barefootnote{\tiny\shortfullcite{Thompson2024,Landis2025phyddle}}
\end{frame}


% Methods
%   Simulation model
%   Basic ML methods

\begin{frame}
    \frametitle{phyddle analyses}

    % \begin{adjustwidth}{-1em}{-1em}

    \begin{itemize}
        \setlength{\itemindent}{6em}
        \item Simulated 100k, trees (95k training, 5k holdout) with 500 extant tips
        \begin{itemize}
            \setlength{\itemindent}{6em}
            \item Extinct lineages removed
        \end{itemize}
        \item Speciation rate, $\speciationrate \sim \textrm{\sffamily Log-uniform}(0.01, 1)$
        \item Extinction rate $ \sim \textrm{\sffamily Log-uniform}(0.01\speciationrate, \speciationrate)$
        \item Two character states
        \begin{itemize}
            \setlength{\itemindent}{6em}
            \item Transition rate $ \sim \textrm{\sffamily Uniform}(0.1\speciationrate, 0.7\speciationrate)$
        \end{itemize}
    \end{itemize}

    \begin{columns}[T]
        \column{0.48\textwidth}
        \begin{center}
            \textbf{\textit{Parameter estimation}}
        \end{center}
        \begin{itemize}
            \item Burst rate $ \sim \textrm{\sffamily Uniform}(0.3\speciationrate, 1.2\speciationrate)$
            \item State 0 burst prob $ \sim \textrm{\sffamily Uniform}(0.65, 1)$
            \item State 1 burst prob $ \sim \textrm{\sffamily Uniform}(0.1, 0.65)$
        \end{itemize}

        \column{0.48\textwidth}
        \begin{center}
            \textbf{\textit{Model choice}}
        \end{center}
        \begin{itemize}
            \item Burst rate $ \sim \textrm{\sffamily Uniform}(0.2\speciationrate, \speciationrate)$
            \item State 0 burst prob $ \sim \textrm{\sffamily Uniform}(0.6, 1)$
            \item State 1 burst prob = State 0 \\ OR $ \sim \textrm{\sffamily Uniform}(0.0, 0.3)$
        \end{itemize}
    \end{columns}
    % \end{adjustwidth}
\end{frame}


% Results of simple birth-death-burst analyses
% Results of 2 state-dep burst rates
% Results of model choice between 1 and 2 burst rates

% \begin{frame}[t]
    \frametitle{Results: Simple BD+Burst model}
    \begin{columns}[c]
        \column{.33\textwidth}

        \begin{minipage}[t][\frametextheight][t]{\columnwidth}
            \begin{center}
                % Speciation rate
                \includegraphics[width=\textwidth,height=\frametextheight,keepaspectratio]{../../bdb-simple/custom-plots/out.test_estimate_log10_birth_rate_0-cropped.pdf}
            \end{center}
        \end{minipage}

        \column{.33\textwidth}

        \begin{minipage}[t][\frametextheight][t]{\columnwidth}
            \begin{center}
                % Exctinction rate
                \includegraphics[width=\textwidth,height=\frametextheight,keepaspectratio]{../../bdb-simple/custom-plots/out.test_estimate_log10_death_rate_0-cropped.pdf}
            \end{center}
        \end{minipage}

        \column{.33\textwidth}

        \begin{minipage}[t][\frametextheight][t]{\columnwidth}
            \begin{center}
                % Burst rate
                \includegraphics[width=\textwidth,height=\frametextheight,keepaspectratio]{../../bdb-simple/custom-plots/out.test_estimate_log10_expected_burst_rate_0-cropped.pdf}
            \end{center}
        \end{minipage}

    \end{columns}
\end{frame}

\begin{frame}
    \frametitle{Results: Number of burst events}

    \begin{columns}[T]
        \column{0.49\textwidth}
        \begin{minipage}[c][0.75\frametextheight][c]{\columnwidth}
            \begin{center}
                % \textbf{\textit{Parameter estimation}}

                \includegraphics[width=\columnwidth,height=0.8\frametextheight,keepaspectratio]{../../bdb-state-dep-continuous/custom-plots/burst-with-extant-nodes-counts.pdf}
            \end{center}
        \end{minipage}

        \column{0.49\textwidth}
        \begin{minipage}[c][0.75\frametextheight][c]{\columnwidth}
            \begin{center}
                % \textbf{\textit{Model choice}}

                \includegraphics[width=\columnwidth,height=0.8\frametextheight,keepaspectratio]{../../bdb-state-dep/custom-plots/burst-with-extant-nodes-counts.pdf}
            \end{center}
        \end{minipage}

    \end{columns}

    \begin{center}
        \uncover<2->{
        \textbf{\textit{The simulated trees do not have large numbers of burst events to learn from}}
        }
    \end{center}
\end{frame}

\begin{frame}[t]
    \frametitle{Results: State-dependent burst model choice}

    \vspace{-3.5mm}
    \begin{center}
        % \begin{minipage}[t][0.54\frametextheight][t]{\textwidth}
            \uncover<2->{
                \includegraphics[width=0.35\textwidth,height=0.55\frametextheight,keepaspectratio]{../../bdb-state-dep-continuous/custom-plots/out.test_estimate_log10_expected_burst_rate_0-cropped.pdf}
                \hspace{3mm}
                \includegraphics[width=0.35\textwidth,height=0.55\frametextheight,keepaspectratio]{../../bdb-state-dep-continuous/custom-plots/out.test_estimate_log10_expected_burst_rate_diff-cropped.pdf}
            }
        % \end{minipage}

        % \begin{minipage}[t][0.54\frametextheight][t]{\textwidth}
                \includegraphics[width=0.35\textwidth,height=0.55\frametextheight,keepaspectratio]{../../bdb-state-dep-continuous/custom-plots/out.test_estimate_log10_birth_rate_0-cropped.pdf}
                \hspace{3mm}
                \includegraphics[width=0.35\textwidth,height=0.55\frametextheight,keepaspectratio]{../../bdb-state-dep-continuous/custom-plots/out.test_estimate_log10_death_rate_0-cropped.pdf}
                \hspace{3mm}
                \includegraphics[width=0.35\textwidth,height=0.55\frametextheight,keepaspectratio]{../../bdb-state-dep-continuous/custom-plots/out.test_estimate_log10_state_rate_01-cropped.pdf}
        % \end{minipage}
    \end{center}

\end{frame}

\begin{frame}[t]
    \frametitle{Results: State-dependent burst \textbf{\textit{model choice}}}

    \vspace{-4.5mm}
    \begin{columns}[T]
        \column{.33\textwidth}

        \begin{minipage}[t][0.54\frametextheight][t]{\columnwidth}
            \begin{center}
                % Speciation rate
                \includegraphics<2->[width=\columnwidth,height=0.54\frametextheight,keepaspectratio]{../../bdb-state-dep/custom-plots/out.test_estimate_log10_expected_burst_rate_0-cropped.pdf}
            \end{center}
        \end{minipage}

        \vspace{1mm}
        \begin{minipage}[t][0.54\frametextheight][t]{\columnwidth}
            \begin{center}
                % Speciation rate
                \includegraphics[width=\columnwidth,height=0.54\frametextheight,keepaspectratio]{../../bdb-state-dep/custom-plots/out.test_estimate_log10_birth_rate_0-cropped.pdf}
            \end{center}
        \end{minipage}

        \column{.33\textwidth}

        \begin{minipage}[t][0.54\frametextheight][t]{\columnwidth}
            \begin{center}
                % Speciation rate
                \includegraphics<2->[width=\columnwidth,height=0.54\frametextheight,keepaspectratio]{../../bdb-state-dep/custom-plots/out.test_estimate_log10_expected_burst_rate_1-cropped.pdf}
            \end{center}
        \end{minipage}

        \vspace{1mm}
        \begin{minipage}[t][0.54\frametextheight][t]{\columnwidth}
            \begin{center}
                % Exctinction rate
                \includegraphics[width=\columnwidth,height=0.54\frametextheight,keepaspectratio]{../../bdb-state-dep/custom-plots/out.test_estimate_log10_death_rate_0-cropped.pdf}
            \end{center}
        \end{minipage}

        \column{.33\textwidth}

        \begin{minipage}[t][0.54\frametextheight][t]{\columnwidth}
            \begin{center}
                % Speciation rate
                \includegraphics<2->[width=\columnwidth,height=0.54\frametextheight,keepaspectratio]{../../bdb-state-dep/custom-plots/out.test_estimate_model_type-cropped.pdf}
            \end{center}
        \end{minipage}

        \vspace{1mm}
        \begin{minipage}[t][0.54\frametextheight][t]{\columnwidth}
            \begin{center}
                % Burst rate
                \includegraphics[width=\columnwidth,height=0.54\frametextheight,keepaspectratio]{../../bdb-state-dep/custom-plots/out.test_estimate_log10_state_rate_01-cropped.pdf}
            \end{center}
        \end{minipage}

    \end{columns}
\end{frame}

\begin{frame}[c]
    \frametitle{Results: State-dependent burst \textbf{\textit{model choice}}}

    \begin{center}
        \includegraphics[width=0.48\textwidth,height=0.82\frametextheight,keepaspectratio]{../../bdb-state-dep/custom-plots/out.test_estimate_model_type-cropped.pdf}
        \hspace{6mm}
        \includegraphics[width=0.48\textwidth,height=0.82\frametextheight,keepaspectratio]{../../bdb-state-dep-continuous/custom-plots/out.test_estimate_root_state-cropped.pdf}

        \bigskip
        \textbf{\textit{Estimating burst model better than ancestral state**}}
    \end{center}

\end{frame}


\begin{frame}
    \frametitle{Take-homes}
    
    Trees generated by a BD model with state-dependent burst rates have
    information about those burst rates
    
    \bigskip
    phyddle is a user-friendly tool for quickly experimenting with new
    (hair-brained) phylogenetic models

    \bigskip
    Caveats:
    \begin{itemize}
        \item Analyses assumed no model violations
        \item Trees aren't observable!
    \end{itemize}
\end{frame}

% Caveats
%   Data aren't trees!
%   No model violations

% Future directions
%   Intro polytomies into burst events
%       Burst-y processes can cause them
%   Dev Bayesian implementation of state-dep shared div models
%       Learn from the actual data
\begin{frame}
    \frametitle{Next steps}

    Introduce polytomies into burst events
    \begin{itemize}
        \item Burst-y processes predict them
        \begin{itemize}
            \item \Eg, Do rising sea levels always split one island into two?
            \item \Eg, Does a carrier always infect one other individual at a
                social gathering?
        \end{itemize}
    \end{itemize}

    \bigskip
    \uncover<2->{
    Develop full Bayesian implementation of state-dependent shared divergence models
    \begin{itemize}
        \item Learn from actual data (sequences) while integrating phylo uncertainty
        \item Can make probability statements about events within the tree
            \begin{itemize}
                \item The neural network is trained on random trees
            \end{itemize}
    \end{itemize}
    }
\end{frame}


\begin{frame}
    \frametitle{Open science: everything is available\ldots}
    \begin{columns}[c]
        \column{.45\textwidth}

        \begin{minipage}[c][\frametextheight][c]{\columnwidth}
            \begin{center}

            \textbf{\textit{Simulator:}}

            \includegraphics[height=0.55\frametextheight,width=0.8\columnwidth,keepaspectratio]{../images/qr-code-SDSDsim.png}
            \captionof{figure}{\href{https://github.com/phyletica/SDSDsim}{\small github.com/phyletica/SDSDsim}}

            \end{center}

        \end{minipage}

        \column{.54\textwidth}

        \begin{minipage}[c][\frametextheight][c]{\columnwidth}
            \begin{center}

            \textbf{\textit{Open-science notebook:}}

            \includegraphics[height=0.55\frametextheight,width=0.8\columnwidth,keepaspectratio]{../images/qr-code-SDSDsim-phyddle-experiments.png}
            \captionof{figure}{\href{https://github.com/phyletica/SDSDsim-phyddle-experiments}{\small github.com/phyletica/SDSDsim-phyddle-experiments}}

            \end{center}

        \end{minipage}

    \end{columns}
\end{frame}


% Shout-outs
%   Phyddle (Ammon and Michael)
%       Very user-friendly; responsive to issues
%   Paul Lewis C++ tutorial???
\begin{frame}{Shout-outs}
    \begin{columns}[t]
        \column{.49\textwidth}
            % {\bf Ideas and feedback:}
            \begin{myitemize}
                \item \href{https://phyletica.org}{Phyletica Lab} (the Phyleticians)
                \item Perry Wood, Jr.
                \item Ammon Thompson
                \item Michael Landis
                \item Sebastian H\"{o}hna
            \end{myitemize}

            \smallskip
            {\bf Computation:}\\
            % \includegraphics<1->[height={8mm}]{../images/au.pdf}
            \begin{myitemize}
                \item Alabama Supercomputer Authority
                \item Auburn University Easley Cluster
            \end{myitemize}

        \column{.49\textwidth}

            {\bf Photo credits:}
            \begin{myitemize}
                % \item Rafe Brown
                \item Perry Wood, Jr.
                \item \href{http://phylopic.org/}{PhyloPic}
            \end{myitemize}

            \smallskip
            % {\bf Funding:}\\
            \begin{tabular}{@{}m{25mm}m{25mm}@{}}
            % \includegraphics<1->[height={15mm}]{../images/nsf.jpg} & \\% DEB 1656004 \\
            \includegraphics<1->[height={25mm}]{../images/au.jpg} &
            \includegraphics<1->[height={25mm}]{../images/aumnh-crop-thumb.png} \\
            \end{tabular}
    \end{columns}
    
\end{frame}


\begin{frame}
    \frametitle{Questions?}    
    \begin{columns}[c]
        \column{.51\textwidth}

        \begin{minipage}[c][\frametextheight][c]{\columnwidth}
        \begin{center}
            {
            \Large
            \href{mailto:joaks@auburn.edu}{joaks@auburn.edu}

            % \bigskip
            % \href{http://phyletica.org/}{phyletica.org}
            }

            \bigskip
            \begin{center}
                \textbf{Slides}:\\
                \href{https://phyletica.org/slides/evol2025.pdf}{%
                \includegraphics[height=0.4\headlessframetextheight]{../images/qr-code-evol2025.png}}
                \captionof{figure}{\href{https://phyletica.org/slides/evol2025.pdf}{\normalsize phyletica.org/slides/evol2025.pdf}}
            \end{center}
        \end{center}
        \end{minipage}

        \column{.47\textwidth}

        \begin{minipage}[t][\frametextheight][b]{\columnwidth}
            \begin{figure}
                \begin{center}
                    \smartgraphic{}{../images/darwin-tol-copyright-boris-kulikov-2007.jpg}
                \vspace{-2.0mm}
                \caption{\tiny \copyright~2007 Boris Kulikov \href{https://boris-kulikov.blogspot.com/}{boris-kulikov.blogspot.com}}
                \end{center}
            \end{figure}
        \end{minipage}

    \end{columns}
\end{frame}



% Acknowledgments

% \begin{frame}
% \frametitle{Outline}
% \tableofcontents
% \end{frame}

% \begin{frame}
% \frametitle{Outline}
% \tableofcontents[currentsection,currentsubsection]
% \end{frame}

% Extra slides
%%%%%%%%%%%%%%

% \begin{frame}[noframenumbering]
% \end{frame

\end{document}
